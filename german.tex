%%%%%%%%%%%%%%%%%%%%%%%%%%%%%%%%%%%%%%%%%
% Medium Length Professional CV
% LaTeX Template
% Version 3.0 (December 17, 2022)
%
% This template originates from:
% https://www.LaTeXTemplates.com
%
% Author:
% Vel (vel@latextemplates.com)
%
% Original author:
% Trey Hunner (http://www.treyhunner.com/)
%
% License:
% CC BY-NC-SA 4.0 (https://creativecommons.org/licenses/by-nc-sa/4.0/)
%
%%%%%%%%%%%%%%%%%%%%%%%%%%%%%%%%%%%%%%%%%

%----------------------------------------------------------------------------------------
%	PACKAGES AND OTHER DOCUMENT CONFIGURATIONS
%----------------------------------------------------------------------------------------

\documentclass[
	a4paper, % Uncomment for A4 paper size (default is US letter)
	10pt, % Default font size, can use 10pt, 11pt or 12pt
]{resume} % Use the resume class


% \usepackage[familydefault,light]{Chivo} %% Option 'familydefault' only if the base font of the document is to be sans serif
% \usepackage[T1]{fontenc}


% \usepackage[sfdefault]{inter} %% Option 'sfdefault' only if 'inter' is to be
% % the base font of the document
% \usepackage[T1]{fontenc}


% \usepackage[defaultfam,tabular,lining]{montserrat} %% Option 'defaultfam'
% %% only if the base font of the document is to be sans serif
% \usepackage[T1]{fontenc}
% \renewcommand*\oldstylenums[1]{{\fontfamily{Montserrat-TOsF}\selectfont #1}}

% \usepackage{nunito}
% \usepackage[T1]{fontenc}


% \usepackage[default,oldstyle,scale=0.95]{opensans} %% Alternatively
% %% use the option 'defaultsans' instead of 'default' to replace the
% %% sans serif font only.
% \usepackage[T1]{fontenc}

\usepackage[default]{sourcesanspro}
\usepackage[T1]{fontenc}

%------------------------------------------------

\name{Anurag Parida} % Your name to appear at the top

% You can use the \address command up to 3 times for 3 different addresses or pieces of contact information
% Any new lines (\\) you use in the \address commands will be converted to symbols, so each address will appear as a single line.

\address{Wilhelmstraße 10 \\ 97421 Schweinfurt, DE} % Main address

\address{(0176)~568~00863 \\ anuragparida37@gmail.com} % Contact information

\link{linkedin.com/in/anuragparida \\ github.com/anuragparida}
\link{anuragparida.com}

%----------------------------------------------------------------------------------------

\begin{document}

%----------------------------------------------------------------------------------------
%	EDUCATION SECTION
%----------------------------------------------------------------------------------------

\begin{rSection}{Bildung}

	\textbf{Technische Hochschule Würzburg Schweinfurt} \hfill \textit{Oktober 2022 - Mai 2025} \\
	Bachelor in Robotik (Englisch) \\
	Mitglied der Studierenden Vetretung, Startbahn27 Startup Accelerator \\

\end{rSection}

%----------------------------------------------------------------------------------------
%	WORK EXPERIENCE SECTION
%----------------------------------------------------------------------------------------

\begin{rSection}{Berufserfahrung}

	\begin{rSubsection}{Wissenschaftlicher Mitarbeiter, THWS}{October 2023 - Heute}{Anwendungs und Architekturs Entwickler}{Schweinfurt, Deutschland}
		\item Er arbeitete unter Prof. Bernd Ankenbrand, um die E-Learning-Plattform der Universität zu modernisieren und das Verfolgen von Lernzielen für Studierende und Professoren zu erleichtern.
		\item Er erstellte neue Peer-to-Peer-Datenpipelines, aggregierte Leistungsdiagramme und personalisierte Lernvorschläge.
	\end{rSubsection}

	%------------------------------------------------

	\begin{rSubsection}{Zepto (KiranaKart, YC SS21)}{November 2020 - Februar 2021}{Produkt- und Technikleiter, Erster Mitarbeiter}{Mumbai, India}
		\item Er erstellte als erster Mitarbeiter des Gründerteams die grundlegende Systemarchitektur und die benutzerorientierte App für ein Y-Combinator-Startup. Durchführung von Erstnutzer-Interviews zur Geschäftsplan-Entwicklung.
		\item Skalierung der App-Downloads von 0 auf 100K+ Downloads und half bei einer Finanzierungsbeschaffung von 60 Millionen US-Dollar.
		\item Mithilfe der Auswahl des Führungsteams.

	\end{rSubsection}

\end{rSection}

%----------------------------------------------------------------------------------------
%	TECHNICAL STRENGTHS SECTION
%----------------------------------------------------------------------------------------

\begin{rSection}{Fähigkeiten}

	\begin{tabular}{@{} >{\bfseries}l @{\hspace{6ex}} l @{}}
		Fachgebiete & Python, C++, Deep Learning             \\
		Frameworks  & MERN Stack, Kotlin, AWS/GCP/Azure      \\
		Werkzeuge   & Git, Postman, JSON                     \\
		Soft Skills & Führung, Problemlösung, Konfliktlösung
	\end{tabular}

\end{rSection}

%----------------------------------------------------------------------------------------
%	Languages SECTION
%----------------------------------------------------------------------------------------

\begin{rSection}{Sprachkenntnisse}

	\begin{tabular}{@{} >{\bfseries}l @{\hspace{6ex}} l @{\hspace{6ex}} >{\bfseries}l @{\hspace{6ex}} l @{}}
		Englisch   & Muttersprache (C2) & Deutsch & Mittelstufe (B2) \\
		Hindi/Urdu & Muttersprache      & Odia    & Muttersprache    \\
	\end{tabular}

\end{rSection}

%----------------------------------------------------------------------------------------

\begin{rSection}{Auszeichnungen und Ehrungen}

	\begin{achSubsection}{Google Code-in Gewinner + Mentor für Google Summer of Code}{Oktober 2018 - August 2021}
		\item Wurde unter 54 Gewinnern von 1800 Teilnehmern für den jährlichen Programmierwettbewerb von Google für Schüler ausgezeichnet. Wurde zu einem vollständig bezahlten Besuch zum Google-Hauptsitz in Mountain View eingeladen.
	\end{achSubsection}

	%------------------------------------------------

	% \begin{achSubsection}{Gewinner der CBSE National Science Fair}{Januar 2020}
	% 	\item Projekt 'ALM8' - Alzheimer's Mate wurde als Gewinner unter 10000+ Schulen in Indien in der Endrunde der Kategorie 'Assistive Technology' ausgewählt.
	% \end{achSubsection}

	%------------------------------------------------

	\begin{achSubsection}{Grand Finalist bei HackHarvard MLH von Alibaba Tianchi}{Januar 2019}
		\item Wurde von Alibaba Tianchi eingeladen, mein Projekt 'ALM8 - Alzheimer's Mate' an der Harvard University während der MLH-Woche 2019 zu präsentieren. Das Projekt wurde unter die Top 3 von 500+ offenen Teilnehmern auf der Plattform 'Codechef' gewählt.
	\end{achSubsection}

	%------------------------------------------------

	\begin{achSubsection}{International Olympiad in Linguistics Camp von Microsoft}{Mai 2019}
		\item Wurde unter 30 Schülern aus ganz Indien ausgewählt, um an einem 14-tägigen Trainingslager am IIIT Hyderabad, Indien, teilzunehmen, das von Microsoft in den Bereichen Computer- und Soziolinguistik, NLP und Morphologie veranstaltet wurde.
	\end{achSubsection}

\end{rSection}

\begin{rSection}{Hobbys}

	Reisen, Bouldern, Snooker, Cricket, Kabaddi


\end{rSection}

\end{document}
